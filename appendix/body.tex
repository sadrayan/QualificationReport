% \appendix
\begin{appendices}
\chapter{Related Work}

\section{Population Displacement Factors  - Agent-Based Modeling Approach}
The population forced displacement and migration is a critical issue that has grappled many nations around the world. As the conflicts brewing in many regions of the world, populations are forced to move within, and beyond the international borders. The rising number of the displaced population has challenged many countries and the international community to understand the root causes and factors and improve the humanitarian efforts required to handle the mass populations moving across the borders. 

The causes forced displacement is not limited to the armed conflict, and it can be attributed to many factors such as political and economic instability, complex emergencies, climate and environmental change. Davenport et al. identified the threats to personal safety and security to the leading factor for forced displacement \cite{davenport2003}. The immediate threat to the security is a particularly important factor in regions with armed conflicts, such as Iraq and Syria. In this research, we focus on the population displacement factors leading to mass population displacement in the Middle East region. 

The population displacement has been traditionally considered a political and social issue, with less focus on the how the computational research and enrich the decision-making process. Agent-Based Modeling (ABM) has emerged as an effective method to model the complex social and environmental issues. Work done by \cite{Sokolowski2014} focuses on the modeling and simulating population displacement in the Syrian city of Aleppo. After identifying the prompting factors leading to displacement, an agent-based model is devised. The agents (i.e. persons) in the agent-based model can dynamically coordinate and impact other agents. The use of ABM as a means of characterizing the individuals and entities enables the researchers to simulate the historical data and evaluate the results. The model simulates the refugees' behavior in the areas of conflict by capturing the dynamic between the agents involved in the conflict. Furthermore, the simulations can be used to prepare and anticipate the flow of the population across the geographical boundaries.

Another attempt at the ABM in the context of population displacement is made by \cite{asgary2016}. In order to analyze the events and factors leading to forced displacement, an ABM simulation model is proposed. The agents are characterized by the most relevant actors in the conflict in Iraq. The agents identified range from militia insurgents, families, Iraqi Security Forces(local government), and international coalition forces. Agents are furthermore characterized by the parameters that govern their behavior. The population displacement data is retrieved from International Organization for Migration (IOM) \footnote{International Organization for Migration - IOM  \url{https://www.iom.int/}.}, which tracks the population movements in the region, including Syria and Iraq.  Integrating the ABM and Geographic Information System (GIS), the simulations is generate using AnyLogic \footnote{AnyLogic Simulation Software \url{https://www.anylogic.com/}}  simulation software. 

The initial decision of the agents (i.e. families) is dependent on the values of Risk Index in the geographical proximity. Risk index, as defined in \cite{asgary2016}, is a function of a number of dynamic variables. These variables may have pull (i.e. positive, neutral) or push (i.e. negative, escape) effect on the displacement. The risk index is affected by the variables such as:

\begin{itemize}
    \item presence of national military (Iraqi Security Forces)
    \item presence of coalition forces 
    \item presence of militia forces (i.e. ISIS forces)
    \item access to humanitarian aid
    \item access to health care
    \item access to transportation infrastructure
    \item economical stability
\end{itemize}

As the risk index values change, the pull/push signals fluctuate, and the population dynamic changes. The higher the Risk Index in the region (i.e. governorates of Iraq), the greater push factor, and ultimately population displacement. In this research, we take expand on the risk index factors to derive the push/pull factors. The risk factors can be aligned with the overall topics in the news media to estimate the push/pull signal. 


\section{Background}

\subsection{Population Displacement Factors}
In this research, we investigate the possibilities of digesting a large news media corpus to analyze the factors of forced population displacement in conflict zones. The sheers amount and diversity of available news media has provided an opportunity for the researchers to investigate the relationship between the media and the movement of the populations. This dynamic of media content and the population behavior is complex as the societies and means of communications becomes more diverse. The role and power of the media over the direction of the societies is a universal challenge that each society should engage. 

The crisis of the mass population displacement has taken a new turn as the populations are pushed out of conflict-torn regions. The issue of forced population displacement is not a new phenomenon in human history.  There are different types of population displacement, which requires an understanding of different causes, and how the humanitarian efforts can be deployed to alleviate the crisis.  First overall categories are the forced versus voluntary population displacements. The voluntary population displacement or immigration has been an integral part of many multi-cultural nations. However, with the unprecedented influx of forcibly displaced populations, many of the host nation’s immigration policies are overwhelmed. The focus of this research is on the forced population displacement. 

We can categorize the forced population displacement based on the population is dispersed in geographical regions. Internally displaced people (IDPs) don’t cross the borders to find safety. While IDPs may have fled for similar reasons as refugees, they stay within their own country and remain under the protection of their government, even if that government is the reason for their displacement. As a result, these people may be among the most vulnerable in the world. The refugees, on the other hand, are the people who are outside the country of their nationality. In most cases, the refugees seek safety in the neighboring countries. In recent years, the number of refugees seeking shelter in across international borders has sored. According to UN high commission for refugees, Middle East, North Africa and Sub-Saharan Africa generated large-scale population displacement and hosted the majority of refugees worldwide. The Syrian civil war has pushed many civilians out of the country, and many of those are seeking asylum and refugee status at the doorsteps of Europe. 

In this work, we take a look at the content of the news media as a textual representation of the events in the real word. Taking this straightforward approach, we aim at categorizing the news media by the relevant topics to identify the factors leading to population displacements across conflict zones. The goal is to identify a  set of overall topics across the news media corpus and aligned them with the forced displacement factors. The conflict region we focus on is the Middle East, with particular attention to the Syrian crisis and security instabilities in Iraq. The key factors identified in this context include, but not limited to, the following:

\begin{itemize}
    \item Terrorism (i.e. Daesh/ISIS). The presence of terrorist groups, namely ISIS, the region has been one of driving forces of population displacement in the region. The threat of imminent violence is the major decision making factor for the individuals and families who are situated in proximity of the extremist groups.
    
	\item Environmental factors: there's a growing body of literature which focuses on the environment as a driver, determinant or trigger causing the population displacements \cite{Greiner2016}. The environmental factors may range from immediate threats, such as storm or wildfire, to slow onset disaster events such as droughts and desertification \cite{Morrissey2009}. 

    \item  Political interference and lack of effective government. The government has a diverse set of roles in a country, ranging from setting internal agenda to internal affairs. The government has a responsibility to provide basic means of life for the cities (i.e. access to security, transportation, energy, education, and health). Incapacity of the local government to ensure these basic needs of the society is identified as a cause of population displacement,

    \item  Economic Issues. The economic security and stability is another important population displacement factor. The lack of economic opportunity and lack of access to basic needs of life (such as food, water, health) are among the deciding factors in the displacement of the population. The economic factors may be affected by the political agenda set by the local government. The environmental factors also play a major role in the economic well-being of the societies faces challenges of the climate change and its effects in widespread droughts and rising sea levels. 
    
	\item Religious, ethnic and historical differences. The power of religion and ethnic belonging is another source of displacement. A variety of ethnic and religious groups are actively engaged in the conflict. These active players range from different sects of Islam (Sunni, Shia), to the ethnic groups such as Kurds, Peshmerga, and Yezidis.


\end{itemize}

Inspired by the works done in computational and social sciences, the overarching objective of this work is to bring insight into the causes and signals of population displacement across different regions of the world. We conjecture that the forces driving the population displacement can be correlated with the overall topics discussed in the news media pertaining the region. In order to move towards this conjecture, we perform an extensive analysis of topics in a large news media corpus to identify and align the displacement factors and the topics discussed in the corpus. Preliminary results show a degree of alignment between the identified population displacement factors and the topics discovered from the news media corpus. 



\end{appendices}