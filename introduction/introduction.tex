\chapter{Introduction}

\section{Motivation and Objectives}
The motivation behind this report is to investigate various natural language analysis techniques to establish a strong understanding of the available methods. In order to achieve this goal, we take a look into the natural language processing and generation methods that are used in creating dialogue systems. Second, we analyze the classification tasks and it's application to fake news detection. Lastly, topic modelling and analysis is applied to large news media to extract violence-related topics, which can be used in population displacement prediction. These areas of research can guide my future research.

\section{Contributions}
The contributions of this report are as follows:
 
\begin{itemize}
    \item Overview of the state-of-the-art natural language generation techniques and it's application in creating a political debate dialogue system. The extensive experiments show that neural network based generative models can produce dialogue responses. 
    \item In an inquiry into fake news phenomena, we describe the characteristics of the various dataset and evaluate the classification task performance to set a strong baseline. The goal of this section is to introduce the datasets used by researchers and point to their shortcomings to be a representative sample of the real phenomena. As well, we set a strong classification baseline that can be helpful for the researchers in the field.
    \item Lastly, we analyze the topic modelling of the news media corpus to extract the violence related signals for population displacement. The dynamic topic modelling for the monthly period shows promising results for extractions of displacements signals that improve population displacement prediction. Experiments show that the high coherence topics are a good indicator of rising and fall of violence, which contributes to the population movements.
\end{itemize}
