\chapter{Conclusion}

\label{ch:conclusions}

\section{Summary of Report Achievements}
In chapter 1, we present an interactive dialogue system that allows users to interact with a chatbot that is modelled on the political debates. This work includes natural language processing and generation techniques to generate dialogue responses to the user input. The goal of this work was to utilize the state-of-the-art generative neural networks and train them on political debate dataset and evaluate how the generative models operate in this domain-specific context. In chapter 2 we took an inquiry in understanding the fake news phenomena and evaluated various classification techniques on multiple available datasets. The fake news has become a serious issue in political discourse, and yet it seems that the science to combat it has remained behind. The fake news classification task shows average performance, however, the availability of the representative dataset remains as an obstacle to combat fake news.

Lastly, we take a look at an effective topic modelling approach to analyze the news media to extract population displacements signals. We evaluate different topic modelling techniques and evaluate the topic coherence measures to find the best number of topics for a particular period. Based on this initial work, we further show that the violence related signals can be extracted from the dynamic topics, and can improve the population displacement prediction. 


\section{Future Plans}
Overall, we this report focuses on the text analysis techniques, ranging from topic modelling, text classification, and text generation. These are three active research topics that open various lines of research for the future of my academic research. In future, I'm planning to expand on natural language analysis and modelling in the political science context. This work is encouraged by the availability of large political related data produced and available from government institutions and agencies. Analysis of the legislative voting rolls based on the content of the bills and the voting history of the legislator is an active research area that can benefit from state-of-the-art natural language understanding and classification techniques. 